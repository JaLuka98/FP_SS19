\section{Diskussion}
\label{sec:Diskussion}

Bei der Durchführung dieses Versuchs werden signifikante Anstrengungen unternommen, eine möglichst genaue Messung vorzunehmen.
Die Probe ist von einem Zylinder gleichen Materials umgeben, um Verluste durch Wärmestrahlung zu minimieren.
Allerdings reagieren die Messgeräte ein wenig träge, sodass die Temperaturen der beiden Objekte nur im Rahmen einer gewissen Abweichung gleich sind. Außerdem kann kein perfektes Vakuum hergestellt werden, sodass weitere Verluste entstehen.

Die Abweichung des experimentellen vom theoretischen Wert von $-12.48\%$ ist erscheint jedoch zu groß, um nur durch diese Effekte erklärt zu werden. Es muss auch in Betracht gezogen werden, dass die Theorieformeln für die Debye-Temperatur und insbesondere für die Debye-Frequenz Näherungen entstammen. Außerdem werden die Schallgeschwindigkeiten als fehlerlos angesehen, sodass der Theoriewert keine Unsicherheit aufweist.\\
Desweiteren werden auch nicht im kompletten Temperaturbereich Messungen genommen. Dies kann die Genauigkeit der Messung insbesondere deswegen negativ beeinflussen, da das Debye-Modell insbesondere nahe $\SI{0}{\kelvin}$ besonders genau ist, während es im mittleren Bereich durchaus Abweichungen von experimentell gemessenen Werten aufweist, da auch im Debye-Modell Näherungen vorgenommen werden.

Im Hochtemperaturlimes erscheinen die bestimmten Werte für die Wärmekapazität korrekt,
da sie um den Dulong-Petit-Wert von $3R \approx \SI{24.94}{\joule\per\mole\kelvin}$ streuen.
Dies ist in Abbildung \ref{fig:cv} erkennbar.

Außerdem sind die Werte für die Widerstände $R$ und die Stromstärken $I$ hier als
Fehlerlos angenommen. In Wirklichkeit weisen diese Werte aber auch Fehler auf. Somit ist dauch der Fehler des experimentell bestimmten Werts unterschätzt.

Ingesamt kann die Durchführung des Versuchs als erfolgreich bewertet werden.
Die recht große Abweichung vom theoretischen Wert könnte durch eine verbesserte Reduktion des Wärmeverlusts der Probe reduziert werden.
Eine erneute Durchführung des Versuchs könnte ebenso einen genaueren Wert ergeben, da die Handhabung der elektronischen Geräte während des Versuchs erst geübt werden musste.

%kleine verluste durch wärmestrahlung und wärmelietung weil die Temperaturen nicht immer
%exakt gleich waren.
%
%eventutell auch minimal konvektion weil kein komplett perfektes Vakkuum. Aber nah
%dran und deswegen vermutlich sehr kleine effekte.
%
%dadurch dass die ganze zeit stickstoff verdampft kann die Umgebubgstemperatur
%nicht konstant gehalten werden, sodass es auch in der Temperatur der Probe Schwankungen
%gibt.
%
