\section{Auswertung}
\label{sec:Auswertung}
Die Berechnungen werden mit Python, Version 3.6.1, durchgeführt, unterstüzt durch die Bibliothek Numpy 1.21.1 \cite{numpy}. Grafiken werden mit Matplotlib 2.0.2 \cite{matplotlib} angefertigt.
Ausgleichrechnungen werden mithilfe der Python-Bibliothek Scipy 0.19.0 \cite{scipy} durchgeführt.

\subsection{Bestimmung der isobaren Molwärme $C_p$}
\label{subsec:cp}
Zur Bestimmung der Molwärme $C_p$ müssen zunächst aus den gemessenen
Widerständen die Temperaturen berechnet werden. Diese ergeben sich aus dem
Zusammenhang
\begin{equation*}
  T=0.00134R^2\,\frac{\text{K}}{\Omega^2}+2.296R\,\frac{\text{K}}{\Omega}+30.13\text{K} \,.
\end{equation*}
Dabei ist $T$ die Temperatur und $R$ der Widerstand.\\
Die gemessenen Werte sind in Tabelle \ref{tab:messwerte} zu sehen.
Dargestellt sind die Werte für den Widerstand $R_\text{p}$ des Thermometers an der Probe,
den Widerstand $R_\text{z}$ des Thermometers am Kupferzylinder, die Stromstärke $I$ und
die Spannung $U$, mit denen die Probe erwärmt wurde, sowie die Zeit $t$,
die die Probe bis zur nächsten Messung erwärmt wurde.
Die berechneten Temperaturen sind in Tabelle \ref{tab:cp} dargestellt. Für die
Temperaturen $T_\text{p}$ der Probe werden zudem die Differenzen $\Delta T_\text{p}$ der Temperaturen
in den jeweils aufeinanderfolgenden Messungen berechnet.

\begin{table}[htp]
	\begin{center}
    \caption{Direkt während der Durchführung gemessene Werte.}
    \label{tab:messwerte}
		\begin{tabular}{ccccc}
    		\toprule
			{$R_\text{p}/\Omega$} & {$R_\text{z}/\Omega$} & {$I$/µA} & {$t$/s} & {$U/$V}\\
			\midrule
			22,1 & 22,2 & 167,2 & 150 & 17,7\\
			24,5 & 23,1 & 167,3 & 150 & 17,7\\
			26,6 & 24,9 & 167,8 & 150 & 17,7\\
			28,8 & 29,8 & 168,2 & 150 & 17,7\\
			30,9 & 32,5 & 168,4 & 150 & 17,7\\
			33,2 & 34,6 & 168,6 & 150 & 17,7\\
			35,1 & 36,1 & 168,8 & 150 & 17,7\\
			37,1 & 37,5 & 168,8 & 150 & 17,8\\
			39,0 & 38,9 & 169,0 & 150 & 17,8\\
			40,8 & 40,4 & 169,0 & 150 & 17,8\\
			42,6 & 42,7 & 169,1 & 150 & 17,8\\
			44,4 & 44,7 & 169,2 & 150 & 17,8\\
			46,1 & 46,4 & 169,2 & 150 & 17,8\\
			47,9 & 47,9 & 169,3 & 150 & 17,8\\
			49,5 & 49,3 & 169,4 & 150 & 17,9\\
			51,2 & 51,3 & 169,4 & 150 & 17,9\\
			52,9 & 53,6 & 169,4 & 150 & 17,9\\
			54,6 & 55,4 & 169,5 & 150 & 17,9\\
			56,2 & 56,7 & 169,5 & 150 & 17,9\\
			57,8 & 57,9 & 169,5 & 150 & 17,9\\
			59,3 & 58,9 & 169,6 & 150 & 17,9\\
			60,9 & 60,8 & 169,6 & 150 & 17,9\\
			62,5 & 63,1 & 169,6 & 150 & 17,9\\
			64,1 & 64,6 & 169,6 & 150 & 17,9\\
			65,6 & 65,8 & 169,6 & 300 & 17,9\\
			68,6 & 68,4 & 169,7 & 300 & 17,9\\
			71,7 & 72,6 & 169,7 & 300 & 17,9\\
			74,7 & 75,6 & 169,7 & 300 & 17,9\\
			77,6 & 77,6 & 169,8 & 300 & 17,9\\
			80,5 & 80,1 & 169,8 & 300 & 17,9\\
			83,3 & 83,6 & 169,8 & 300 & 17,9\\
			86,0 & 83,4 & 169,8 & 300 & 17,9\\
			88,7 & 90,2 & 169,9 & 300 & 17,9\\
			91,6 & 92,0 & 169,9 & 300 & 17,9\\
			94,4 & 94,2 & 169,9 & 300 & 17,9\\
			97,1 & 96,4 & 169,9 & 300 & 17,9\\
			99,8 & 99,6 & 169,9 & 300 & 17,9\\
			102,5 & 102,1 & 170,0 & 300 & 17,9\\
			105,1 & 104,9 & 170,0 & 300 & 17,9\\
			107,8 & 107,7 & 170,0 & 300 & 17,9\\
			110,4 & 110,6 & - & -& -\\
		\bottomrule
		\end{tabular}
	\end{center}
\end{table}

Die der Probe zugeführte Wärmeenergie ist eine elektrische Energie und lässt sich daher durch
\begin{equation*}
  E=IUt
\end{equation*}
berechnen. Damit kann die molare spezifische Wärmekapazität (auch Molwärme gennant) bei konstantem Druck $C_p$ über den Zusammenhang
\begin{equation*}
  C_p=\frac{E}{n \Delta T}
\end{equation*}
bestimmt werden. Dabei ist $n=m/M$ die Stoffmenge, wobei $m = \SI{0.342}{\kilo\gram}$ die
Masse der Probe und $M = \SI{63.546}{\gram\per\mole}$ \cite{Molmasse} die Molmasse von Kupfer ist.
Die berechneten Werte von $\Delta T$, $E$ und $C_p$ befinden sich in Tabelle \ref{tab:cp}.

\begin{table}[htp]
	\begin{center}
    \caption{Berechnete Werte für die Temperaturen, die der Probe zugeführte Energie und die
    Molwärme.}
    \label{tab:cp}
		\begin{tabular}{cccc}
		\toprule
			{$T_\text{p}/$K} & {$\Delta T_\text{p}/$K} & {$E$/J} & {$C_p/\frac{\text{mol}}{\text{kg K}}$}\\
			\midrule
			81,53 & 5,66 & 443,92 & 14,57\\
			87,19 & 4,97 & 444,18 & 16,62\\
			92,15 & 5,21 & 445,51 & 15,87\\
			97,37 & 4,99 & 446,57 & 16,63\\
			102,36 & 5,48 & 447,10 & 15,16\\
			107,83 & 4,54 & 447,63 & 18,33\\
			112,37 & 4,79 & 448,16 & 17,40\\
			117,16 & 4,56 & 450,70 & 18,38\\
			121,71 & 4,33 & 451,23 & 19,38\\
			126,04 & 4,33 & 451,23 & 19,35\\
			130,37 & 4,34 & 451,50 & 19,32\\
			134,71 & 4,11 & 451,76 & 20,43\\
			138,82 & 4,36 & 451,76 & 19,25\\
			143,18 & 3,88 & 452,03 & 21,63\\
			147,07 & 4,13 & 454,84 & 20,45\\
			151,20 & 4,14 & 454,84 & 20,41\\
			155,34 & 4,15 & 454,84 & 20,37\\
			159,49 & 3,91 & 455,11 & 21,62\\
			163,40 & 3,92 & 455,11 & 21,58\\
			167,32 & 3,68 & 455,11 & 22,98\\
			170,99 & 3,93 & 455,38 & 21,52\\
			174,93 & 3,94 & 455,38 & 21,48\\
			178,86 & 3,95 & 455,38 & 21,45\\
			182,81 & 3,70 & 455,38 & 22,84\\
			186,51 & 7,43 & 455,38 & 11,39\\
			193,94 & 7,70 & 911,29 & 21,99\\
			201,64 & 7,48 & 911,29 & 22,65\\
			209,12 & 7,25 & 911,29 & 23,35\\
			216,37 & 7,27 & 911,83 & 23,30\\
			223,64 & 7,04 & 911,83 & 24,05\\
			230,68 & 6,81 & 911,83 & 24,87\\
			237,50 & 6,83 & 911,83 & 24,80\\
			244,33 & 7,36 & 912,36 & 23,04\\
			251,69 & 7,13 & 912,36 & 23,79\\
			258,81 & 6,89 & 912,36 & 24,60\\
			265,71 & 6,91 & 912,36 & 24,53\\
			272,62 & 6,93 & 912,36 & 24,46\\
			279,55 & 6,69 & 912,90 & 25,34\\
			286,24 & 6,97 & 912,90 & 24,34\\
			293,21 & 6,73 & 912,90 & 25,20\\
			299,94 & - & - & -\\
		\bottomrule
		\end{tabular}
	\end{center}
\end{table}


\subsection{Berechnung der isochoren Molwärme $C_V$}
\label{subsec:cv}

Nun soll mithile von Gleichung \eqref{eqn:CpminusCV} die Molwärme bei konstantem Volumen $C_V$ bestimmt werden.
Hierfür ist es notwendig, die Temperaturabhängigkeit des thermischen Expansionskoeffizienten
$\alpha$ zu berücksichtigen. In Tabelle \ref{tab:alpha} befinden sich Daten zu diesem.
Um den Fehler bei der Auswahl des Wertes für eine gegebene Temperatur zu reduzieren, wird eine Ausgleichsrechnung durchgeführt, nach der sich kontinuierliche Werte für $\alpha$ berechnen lassen.
Die Funktion für die Ausgleichsrechnung ist durch
\begin{equation*}
  \alpha(T)=\frac{a}{T}+b
\end{equation*}
gegeben. Die Interpolation ist in Abbildung \ref{fig:alpha} dargestellt. Es ergeben sich die Parameter
\begin{align*}
  a&=\SI{-873(4)e-6}{\per\kelvin} \,\\
  b&=\SI{19.411(29)e-6}{} \,.
\end{align*}
Es sei darauf hingewiesen, dass dieses Vorgehen eine effektive Theorie darstellt, die keinen Anspruch darauf erhebt, den Verlauf des thermischen Expansionskoeffizienten in Abhängigkeit der Temperatur tatsächlich korrekt zu beschreiben. Alleine die Ergebnisse, insbesondere die geringen Unsicherheiten der Koeffizienten, rechtfertigen diese Interpolation.

\begin{table}[htp]
	\begin{center}
    \caption{Gegebene Werte für den thermischen Expansionskoeffizienten $\alpha$ in Abhängigkeit von der
    Temperatur \cite{versuchsanleitung}.}
    \label{tab:alpha}
		\begin{tabular}{cc}
		\toprule
			{$T$/K} & {$\alpha/10^{-6}$ K}\\
			\midrule
			70,00 & 7,00\\
			80,00 & 8,50\\
			90,00 & 9,75\\
			100,00 & 10,70\\
			110,00 & 11,50\\
			120,00 & 12,10\\
			130,00 & 12,65\\
			140,00 & 13,15\\
			150,00 & 13,60\\
			160,00 & 13,90\\
			170,00 & 14,25\\
			180,00 & 14,50\\
			190,00 & 14,75\\
			200,00 & 14,95\\
			210,00 & 15,20\\
			220,00 & 15,40\\
			230,00 & 15,60\\
			240,00 & 15,75\\
			250,00 & 15,90\\
			260,00 & 16,10\\
			270,00 & 16,25\\
			280,00 & 16,35\\
			290,00 & 16,50\\
			300,00 & 16,65\\
		\bottomrule
		\end{tabular}
	\end{center}
\end{table}

\begin{figure}
  \centering
  \includegraphics[]{build/alpha.pdf}
  \caption{Auftragung des thermischen Expansionskoeffizienten gegen die Temperatur und Graph der
  Ausgleichsfunktion.}
  \label{fig:alpha}
\end{figure}

Mithilfe dieser Daten kann nun aus Gleichung \eqref{eqn:CpminusCV} die Molwärme $C_V$
berechnet werden. Dabei werden die Werte $V_0=\SI{7.11e-6}{\metre\cubic\per\mole}$
\cite{molvolumen} und $K=\frac{1}{\kappa}=\SI{137.8e9}{\newton\per\metre\squared}$
\cite{kompressionsmodul} verwendet. Es ergeben sich die
in Tabelle \ref{tab:cv} dargestellten Werte. In Abbildung \ref{fig:cv} sind diese Werte
graphisch dargestellt. Auf eine Darstellung der Fehlerbalken wird wegen der sehr
geringen Fehler verzichtet.

\begin{table}[htp]
	\begin{center}
    \caption{Berechnete Werte für die Molwärme $C_V$.}
    \label{tab:cv}
		\begin{tabular}{cc}
		\toprule
			{$C_v/\frac{\text{mol}}{\text{kg K}}$}\\
			\midrule
			14,56599 \pm 0,00004\\
			16,61368 \pm 0,00004\\
			15,86553 \pm 0,00004\\
			16,61918 \pm 0,00004\\
			15,15208 \pm 0,00004\\
			18,32132 \pm 0,00004\\
			17,38576 \pm 0,00004\\
			18,36343 \pm 0,00004\\
			19,36600 \pm 0,00004\\
			19,32578 \pm 0,00004\\
			19,29712 \pm 0,00004\\
			20,40442 \pm 0,00004\\
			19,23091 \pm 0,00005\\
			21,60836 \pm 0,00005\\
			20,42374 \pm 0,00005\\
			20,38409 \pm 0,00005\\
			20,34459 \pm 0,00005\\
			21,59012 \pm 0,00005\\
			21,55093 \pm 0,00006\\
			22,94944 \pm 0,00006\\
			21,48804 \pm 0,00006\\
			21,44919 \pm 0,00006\\
			21,41047 \pm 0,00007\\
			22,80046 \pm 0,00007\\
			11,35195 \pm 0,00007\\
			21,94642 \pm 0,00008\\
			22,60215 \pm 0,00008\\
			23,30638 \pm 0,00009\\
			23,24516 \pm 0,00009\\
			24,00130 \pm 0,00010\\
			24,81682 \pm 0,00011\\
			24,74327 \pm 0,00011\\
			22,97571 \pm 0,00012\\
			23,72419 \pm 0,00013\\
			24,53141 \pm 0,00013\\
			24,45941 \pm 0,00014\\
			24,38780 \pm 0,00015\\
			25,27083 \pm 0,00015\\
			24,26263 \pm 0,00016\\
			25,12684 \pm 0,00017\\
		\bottomrule
		\end{tabular}
	\end{center}
\end{table}

\begin{figure}
  \centering
  \includegraphics[]{build/cv.pdf}
  \caption{Auftragung der Molwärme $C_V$ gegen die Temperatur.}
  \label{fig:cv}
\end{figure}

\subsection{Bestimmung der Debye-Temperatur}
\label{subsec:debye}

Für die Werte bis $T_{\text{max}}=170$\,K wird eine Ausgleichrechnung durchgeführt. Die Ausgleichsfunktion ist durch Gleichung \eqref{eqn:cVDebye} gegeben.
Dabei werden die Werte für $C_V$ wegen ihrer sehr geringen Unsicherheiten als nicht fehlerbehaftet betrachtet. Dies ist in Abbildung \ref{fig:debye} dargestellt. Es ergibt sich damit für die Debye-Temperatur
\begin{equation*}
  \Theta_\text{D}=\SI{291(5)}{\kelvin} \,,
\end{equation*}
mit einer relativen Unsicherheit von $1{,}72\%$.

\begin{figure}
  \centering
  \includegraphics[]{build/debye.pdf}
  \caption{Auftragung der Molwärme $C_V$ gegen die Temperatur im Bereich bis
  $T_{\text{max}}=170$\,K und Graph der Ausgleichsfunktion.}
  \label{fig:debye}
\end{figure}

\subsection{Berechnung der theoretischen Debye-Temperatur}
\label{subsec:debyetheo}

Die Debye-Frequenz und die Debye-Temperatur sollen nun analytisch aus den Eigenschaften des Materials bestimmt werden, um einen Vergleich zur experimentellen Bestimmung im vorherigen Kapitel zu ziehen.

Die Debye-Frequenz wird aus Gleichung \eqref{eqn:debyeFrequenz} bestimmt.
Die Anzahldichte $N/L^3$ wird gemäß der Formel $N_\text{A} \rho / M$ bestimmt, wobei $N_\text{A}$ die Avogadro-Konstante ist,
$\rho = \SI{8960}{\kilo\gram\per\cubic\meter}$ \cite{density} die Dichte und $M = \SI{63.546}{\gram\per\mole}$ \cite{Molmasse} die molare Masse von Kupfer ist.
Gemäß \cite{versuchsanleitung} gilt für die Geschwindigkeiten
\begin{align*}
  v_{\text{long}}&= \SI{4.7}{\kilo\metre\per\second} \,,\\
  v_{\text{trans}}&=\SI{2.26}{\kilo\metre\per\second} \,.
\end{align*}
All diese Werte werden als fehlerlos angesehen. Dies ist als Näherung sehr geringer Unsicherheiten gerechtfertigt, weil diese Materialkonstanten genau gemessen werden können.

Für die Debye-Frequenz ergibt sich
\begin{equation}
  \omega_\text{D} \approx \SI{4.353e13}{\per\second}\,.
\end{equation}
Die Debye-Temperatur beträgt dann
\begin{equation*}
  \theta_\text{D} = \frac{\hbar \omega_D}{k_B} \approx \SI{332.48}{\kelvin} \,.
\end{equation*}
Dieser Wert wird als Theoriewert angesehen. Die Abweichung des experimentell bestimmten Wertes $\SI{291(5)}{\kelvin}$ von diesem beträgt $-12.48\%$.
