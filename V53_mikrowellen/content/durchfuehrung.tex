\section{Durchführung}
\label{sec:Durchführung}

\subsection{Untersuchung der Moden}
\label{subsec:moden}

\subsection{Bestimmung der Wellenlänge und Frequenz}
\label{subsec:frequenz}

\subsection{Bestimmung der Dämpfung des Mikrowellenfeldes}
\label{subsec:dämpfung}

\subsection{Bestimmung des Stehwellenverhältnisses}
\label{subsec:swr}

Das Stehwellenverhältnis wird über drei verschiedene Verfahren bestimmt: Die direkte
Methode, die für kleine und mittlere SWR geeignet ist, die 3dB Methode, die für hohe
SWR verwendet wird und die Abschwächer Methode, die ebenfalls für hohe SWR verwendet
wird.

Bei der
direkten Methode wird die Sonde auf der Leitung in ein Maximum verschoben. Dieser
Wert wird dann auf dem SWR Meter mittels der Verstärkung auf 1 geregelt. Danach
wird mit der Sonde ein Minimum gesucht und der zugehörige Wert vom SWR Meter
abgelesen. Dieses Verfahren wird für Stifttiefen von 0, 3, 5, 7, und \SI{9}{\mili\metre}
durchgeführt.

Bei der 3dB Methode wird die Sonde bei \SI{9}{\mili\metre} Stifttiefe in ein
Minimum gefahren. Daraufhin wird mittels Verstärkung der Zeiger des SWR Meters auf
3dB auf der unteren Skala geregelt. Dann werden links und rechts von dem Minimum
Stellen gesucht, an denen sich ein Vollausschlag einstellt. Die Schlittenpositionen
dieser beiden Stellen werden notiert. Es lässt sich über den Zusammenhang
\begin{equation}
  S=sqrt{1+\frac{1}{\sin^2\left(\frac{\pi(x_1-x_2)}{\lambda_{\symup{g}}}\right)}}
  \label{3dB}
\end{equation}
das Stehwellenverhältnis bestimmen. Dabei sind $x_{\symup{i}}$ die beiden gemessenen
Schlittenpositionen.

Für die Abschwächermethode wird die Sonde bei \SI{9}{\mili\metre} Stifttiefe in ein
Minimum gefahren und das Dämpfungsglied auf \SI{20}{\dezibel} eingestellt. Dann
weren gleichzeitig die Sonde verschoben und die Dämpfung erhöht, sodass sich im
Maximum die gleiche Zeigerposition wie zuvor im Minimum ergibt. Daraus kann dann
das Stehwellenverhältnis mit Hilfe von
\begin{equation}
  S=10^{\frac{A_2-A_1}{20}}
  \label{abschwaecher}
\end{equation}
berechnet werden. Hier sind $A_{\symup{i}}$ die eingestellten Dämpfungen.
