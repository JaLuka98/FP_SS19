\section{Theorie}
\label{sec:Theorie}

\subsection{Grundlagen zu Mikrowellen und Hohlleitern}
\label{subsec:grundlagen}
Mikrowellen sind elektromagnetische Wellen im Frequenzbereich von \SI{300}{\Mhz}
bis \SI{300}{\GHz}. Daher sind sie sehr gut für Hohlleiter, da diese in einem
Frequenzbereich von \SI{1}{\GHz} bis \SI{200}{\GHz} arbeiten. Ein Hohlleiter ist
ein Metallrohr, durch das elektromagnetische Wellen verlustarm geleitet werden können.

Jeder Hohlleiter kann verschiedene Typen von Wellen, auch Moden genannt, leiten.
Die Moden unterscheiden sich in der Verteilung des elektrischen und des magnetischen
Feldes. Man unterscheidet TE-Moden, bei denen das eletkrische Feld transversal zur
Ausbreitungsrichtung der Welle und TM-Moden, bei denen das magnetische Feld transversal
zur Ausbreitungsrichtung schwingt. Ein spezialfall sind TEM-Moden, bei denen sowohl
das elektrische alsauch das magnetische Feld transversal zur Ausbreitungsrichtung
schwingt. Für jede Mode im Hohlleiter gibt es eine untere Grenzfrequenz, unterhalb
der kein Energietransport im Hohlleiter möglich ist. Sie kommt durch die Abmessungen
des Hohlleiters und die damit verbundenen Randbedingungen zustande.

Im freien Raum gilt der Zusammenhang
\equation{}


\subsection{Das Refelxklystron}
\label{subsec:klystron}
