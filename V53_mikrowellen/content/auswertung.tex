\section{Auswertung}
\label{sec:Auswertung}

\subsection{Untersuchung der Moden}
In dem ersten Teil der Auswertung wird die Ausbildung von Moden verschiedener Frequenzen in einem Hohlleiter bei unterschiedlichen Reflektorspannungen untersucht. Die Messdaten sind in \ref{tab:moden} dargestellt; die Größe $A$ bezeichne die Amplitude der Mode. Näherungsweise lassen sich die Moden als Parabeln beschreiben. Aus den drei Punkten ergeben sich folglich für jede untersuchte Mode nicht fehlerbehaftete $a, b, c$ mit
\begin{equation}
  f(x) = ax^2 + bx + c\,,
\end{equation}
wobei der Graph von $f$ die Mode beschreiben soll.
Die aufgetragenen Punkte und Graphen sind in \ref{fig:moden} ersichtlich.

\begin{figure}
  \centering
  \includegraphics{build/modes.pdf}
  \caption{Aufgenomme Datepunkte der Moden und daraus berechnete Parabeln.}
  \label{fig:moden}
\end{figure}
