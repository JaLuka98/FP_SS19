\section{Auswertung}
\label{sec:Auswertung}

\subsection{Untersuchung der Moden}
\label{subsec:auswertungModen}
In dem ersten Teil der Auswertung wird die Ausbildung von Moden verschiedener Frequenzen in einem Hohlleiter bei unterschiedlichen Reflektorspannungen untersucht. Die Messdaten sind in \ref{tab:moden} dargestellt; die Größe $A$ bezeichne die Amplitude der Mode.\\

\begin{table}
\centering
\caption{Messwerte für die Untersuchung der Moden.}
\label{tab:moden}
\begin{tabular}{c c c c c c}
\toprule
& $f/$Mhz & $U_\text{l}$/V  & $U_\text{max}$/V & $U_\text{r}$/V & $A$/mV \\
\midrule
 Erste Mode & 9020 &  0 & 38  & 53  & 30    \tabularnewline
Zweite Mode & 9015 & 60 & 70  & 78  & 50.75 \tabularnewline
Dritte Mode & 9012 & 93 & 108 & 117 & 91.9  \tabularnewline
\bottomrule
\end{tabular}
\end{table}

Näherungsweise lassen sich die Moden als Parabeln beschreiben. Aus den drei Punkten ergeben sich folglich für jede untersuchte Mode nicht fehlerbehaftete $a, b, c$ mit
\begin{equation}
  f(x) = ax^2 + bx + c\,,
\end{equation}
wobei der Graph von $f$ die Mode beschreiben soll.
Die aufgetragenen Punkte und Graphen sind in \ref{fig:moden} ersichtlich.

\begin{figure}[H]
  \centering
  \includegraphics{build/modes.pdf}
  \caption{Aufgenomme Datepunkte der Moden und daraus berechnete Parabeln.}
  \label{fig:moden}
\end{figure}

\subsection{Bestimmung der Wellenlänge und Frequenz}
\label{subsec:auswertungfrequenz}

Im Folgenden sollen einige charakteristische Größen der vorliegenden Mikrowelle im Hohlleiter bestimmt werden.
Dazu wurde der Hohlleiter mit einer Schieblehre vermessen. Seine als fehlerlos angenommenen Abmessungen betragen $a = \SI{23}{\milli\meter}$ und $b = \SI{10.2}{\milli\meter}$. Es lässt sich dann für $m=1$ und $n=0$ Gleichung \eqref{eqn:hohlleiterc} zu $\lambda_\symup{c} = 2a$ nähern, sodass sich die Grenzwellenlänge zu $\SI{46}{\milli\meter}$ ergibt.\\
% Wir machen es mal ohne Tabelle, sind ja nur vier Werte.
Die gemessenen Minima waren an den Positionen (in Millimetern) $54{,}4$; $79{,}9$; $104{,}5$ und $128{,}6$ aufzufinden. Die Wellenlänge ist gerade als Abstand zweier Punkte definiert, die in Ordinatenwert und Steigung übereinstimmen, sodass der Abstand zweier Minima gerade die Wellenlänge ist. Eine Mittelung der Differenzen der Stellen ergibt somit für die Wellenlänge im Hohlleiter einen Wert von $\lambda_\symup{g} = \SI{24.7}{\milli\meter}$ mit einer relativen Unsicherheit von $1{,}62\%$.\\
Die Wellenlänge im freien Raum bestimmt sich dann gemäß $\lambda_0 = \frac{1}{\sqrt{ \frac{1}{\lambda_\symup{g}^2} + \frac{1}{(2a)^2}}} = \SI{21.78(28)}{\milli\meter}$, die relative Unsicherheit beträgt $1{,}29\%$.\\
Die Frequenz im freien Raum ergibt sich durch Multiplikation von $\lambda_0$ mit der Lichtgeschwindigkeit im Vakuum $c_0 \approx \SI{3e11}{\milli\meter\per\second}$ zu $f_\symup{exp} = \SI{13770(18)}{\mega\hertz}$, wohingegen die Dip-Methode einen Wert von $f_\symup{dip} = \SI{9006}{\mega\hertz}$ ergibt. Daraus folgt ein relativer Fehler von $-52{,}90\%$.\\
Die Phasengeschwindigkeit der Welle im Hohlleiter ergibt sich zu $v_\symup{ph} = f_\symup{exp} \lambda_\symup{g} = \SI{3.406(13)e+11}{\milli\meter\per\second}$.
%(1.377+/-0.018)e+10$
%9006MHz
%Die gemessenen Positionen der Minima, welche zur Bestimmung der Wellenlänge im Hohlleiter verwendet werden, sind in Tabelle \ref{tab:minima} angegeben.

%\begin{table}
%\centering
%\caption{Positionen der Minima der Mikrowelle im Hohlleiter in Millimetern.}
%\label{tab:moden}
%\begin{tabular}{c c c c}
%\toprule
%1 & 2 & 3 & 4 \\
%\midrule
%54,4 & 79,9 & 104,5 & 128,6 \tabularnewline
%\bottomrule
%\end{tabular}
%\end{table}

\subsection{Bestimmung der Dämpfungskurve}
\label{subsec:auswertungdämpfung}

\begin{figure}
  \centering
  \includegraphics{build/daempfung.pdf}
  \caption{Darstellung der gemessenen Werte sowie der Herstellerangaben
  für die Dämpfung.}
  \label{fig:daempfung}
\end{figure}


\subsection{Bestimmung des Stehwellenverhältnisses}
\label{subsec:auswertungswr}

Die Messwerte zur direkten Messung des Stehwellenverhältnisses sind in Tabelle
\ref{tab:direkt}
dargestellt.

\begin{table}
\centering
\caption{Messwerte für das Stehwellenverhältnis aus der direkten Methode.}
\label{tab:direkt}
\begin{tabular}{c c}
\toprule
Stifttiefe/mm & SWR \\
\midrule
 0 & 1.02 \tabularnewline
 3 & 1.12 \tabularnewline
 5 & 1.57 \tabularnewline
 7 & 3.3 \tabularnewline
 9 & \infty \tabularnewline
\bottomrule
\end{tabular}
\end{table}

Bei der 3dB Methode ergeben sich für eine Stifttiefe von 9mm die beiden Messwerte
\begin{align*}
  x_1=\SI{69.6}{\milli\metre} \,, \\
  x_2=\SI{67.9}{\milli\metre} \,.
\end{align*}

Nach Gleichung \eqref{eqn:3dB} ergibt sich damit für das Stehwellenverhältnis
\begin{align*}
  WERT EINFÜGEN
\end{align*}

Bei der Abschwächermethode wird bei einer Stifftiefe von 9mm und der Einstellung
des Dämpfungsgliedes auf $A_1=\SI{20}{decibel}$ und dem in Kapitel \ref{subsec:swr}
beschriebenen Messverfahren der Wert $A_2=\SI{42}{\decibel}$ aufgenommen. Nach
Gleichung \eqref{eqn:abschwaecher} ergibt sich für das Stehwellenverhältnis
\begin{align*}
  WERT EINFÜGEN
\end{align*}
