\section{Auswertung}
\label{sec:Auswertung}

\subsection{Untersuchung der Moden}
\label{subsec:auswertungModen}
In dem ersten Teil der Auswertung wird die Ausbildung von Moden verschiedener Frequenzen in einem Hohlleiter bei unterschiedlichen Reflektorspannungen untersucht. Die Messdaten sind in \ref{tab:moden} dargestellt; die Größe $A$ bezeichne die Amplitude der Mode.\\

\begin{table}
\centering
\caption{Messwerte für die Untersuchung der Moden.}
\label{tab:moden}
\begin{tabular}{c c c c c c}
\toprule
& $f/$Mhz & $U_\text{l}$/V  & $U_\text{max}$/V & $U_\text{r}$/V & $A$/mV \\
\midrule
 Erste Mode & 9020 &  0 & 38  & 53\tabularnewline
Zweite Mode & 9015 & 60 & 70  & 78\tabularnewline
Dritte Mode & 9012 & 93 & 108 & 117\tabularnewline
\bottomrule
\end{tabular}
\end{table}

Näherungsweise lassen sich die Moden als Parabeln beschreiben. Aus den drei Punkten ergeben sich folglich für jede untersuchte Mode nicht fehlerbehaftete $a, b, c$ mit
\begin{equation}
  f(x) = ax^2 + bx + c\,,
\end{equation}
wobei der Graph von $f$ die Mode beschreiben soll.
Die aufgetragenen Punkte und Graphen sind in \ref{fig:moden} ersichtlich.

\begin{figure}
  \centering
  \includegraphics{build/modes.pdf}
  \caption{Aufgenomme Datepunkte der Moden und daraus berechnete Parabeln.}
  \label{fig:moden}
\end{figure}

\subsection{Bestimmung der Wellenlänge und Frequenz}
\label{subsec:auswertungfrequenz}


\subsection{Bestimmung der Dämpfungskurve}
\label{subsec:auswertungdämpfung}

\begin{figure}
  \centering
  \includegraphics{build/daempfung.pdf}
  \caption{Darstellung der gemessenen Werte sowie der Herstellerangaben
  für die Dämpfung.}
  \label{fig:daempfung}
\end{figure}


\subsection{Bestimmung des Stehwellenverhältnisses}
\label{subsec:auswertungswr}

Die Messwerte zur direkten Messung des Stehwellenverhältnisses sind in Tabelle
\ref{tab:direkt}
dargestellt.

\begin{table}
\centering
\caption{Messwerte für das Stehwellenverhältnis aus der direkten Methode.}
\label{tab:direkt}
\begin{tabular}{c c}
\toprule
Stifttiefe/mm & SWR \\
\midrule
 0 & 1.02 \tabularnewline
 3 & 1.12 \tabularnewline
 5 & 1.57 \tabularnewline
 7 & 3.3 \tabularnewline
 9 & \infty \tabularnewline
\bottomrule
\end{tabular}
\end{table}

Bei der 3dB Methode ergeben sich für eine Stifttiefe von 9mm die beiden Messwerte
\begin{align*}
  x_1=\SI{69.6}{\milli\metre} \,, \\
  x_2=\SI{67.9}{\milli\metre} \,.
\end{align*}

Nach Gleichung \eqref{eqn:3dB} ergibt sich damit für das Stehwellenverhältnis
\begin{align*}
  WERT EINFÜGEN
\end{align*}

Bei der Abschwächermethode wird bei einer Stifftiefe von 9mm und der Einstellung
des Dämpfungsgliedes auf $A_1=\SI{20}{decibel}$ und dem in Kapitel \ref{subsec:swr}
beschriebenen Messverfahren der Wert $A_2=\SI{42}{\decibel}$ aufgenommen. Nach
Gleichung \eqref{eqn:abschwaecher} ergibt sich für das Stehwellenverhältnis
\begin{align*}
  WERT EINFÜGEN
\end{align*}
