\section{Diskussion}
\label{sec:Diskussion}
Die Untersuchung der Moden ergibt ein plausibles Ergebnis, da diese in ähnlicher Form während der Durchführung des Versuchs auf dem Schirm des Oszilloskops zu erkennen waren. Die erste, dunkelblau dargestellte Mode erscheint recht breit. Dies lässt sich dadurch erklären, dass für diese Mode kein stehendes Bild auf dem Oszilloskop erhalten werden konnte, sodass nur ungenau abgelesen werden konnte. Da die Parameter einer Parabel durch drei Punkte fehlerlos sind, führt ein ungenauer Punkt zu einer großen Abweichung in der Gestalt der Parabel. Da bei den anderen Parabeln die Maxima recht zuverlässig abgelesen werden konnten und diese Punkte auch mit den jeweiligen Extremstellen der berechneten Parabeln konsistent sind, lässt sich die begründete Vermutung aufstellen, dass das abgelesene Maximum der dunkelblauen Mode vergleichsweise weit von dem wahren Wert abweicht.\\
Die experimentelle Bestimmung einiger Größen wie Frequenz und verschiedenen Wellenlängen der Mikrowelle im Hohlleiter ist als konsistent zu bewerten. Insbesondere die Frequenzmessung kann dafür herangezogen werden, da mit der Dip-Methode ein genauer Wert ermittelt wurde, der in guter Näherung als wahrer Wert gelten kann. Die Abweichung von $0{,}77\%$ spricht für die Genauigkeit des verwendeten Verfahrens, welches nur den Abstand der Minima bzw. Maxima als Messgrößen benötigt. Die berechnete Phasengeschwindigkeit im Hohlleiter ist größer als die Lichtgeschwindigkeit im Vakuum. Aufgrund der stets mindestens leicht dispersiven Eigenschaften des Hohlleiters stellt dies eine Bestätigung für die Messmethode dar.\\
Das Nachmessen der vom Hersteller angegebenen Dämpfungskurve muss deutlich kritischer bewertet werden. An beiden Diagrammen ist eindeutig zu erkennen, dass die Krümmung der Messwerte nicht mit der exponentiellen Sollkurve übereinstimmt. Auch die Messwerte selbst können quantitativ nicht mit den Sollwerten in Einklang gebracht werden. Obwohl das SWR-Meter teilweise auch stark schwankte und dies die Genauigkeit der Messung sicherlich beeinflusst hat, können diese starken Abweichungen nicht nur dadurch erklärt werden. Vielmehr weist das Messergebnis darauf hin, dass grobe Fehler bei der Versuchsdurchführung vorgelegen haben könnten. Eine gewissenhafte Wiederholung dieser Messung kann Aufschluss darüber geben, ob tatsächlich grobe Fehler für die starken Diskrepanzen verantwortlich sind oder die Methode so ungeeignet zum Nachvollziehen der Dämpfungskurve ist.\\
Die drei Messungen des Stehwellenverhältnis führen zu konsisten Ergebnissen. Der Wert von $\infty$ für das SWR bei der direkten Methode und Stifttiefe von $\SI{9}{\milli\meter}$ ist der Skala des SWR-Meters anzulasten, da diese logarithmisch ist und bereits kurz nach $10$ aufhört und auf $\infty$ springt. Somit ist der Wert grob als $S > 10$ zu verstehen. Der Trend der Messwerte ist plausibel, da ein hohes SWR starke Reflexion bedeutet und dies genau bei großen Stifttiefen auftritt.\\
Die 3dB Methode ergibt für eine Stifttiefe von $\SI{9}{\milli\meter}$ mit ungefähr $9$ ein mit $S > 10$ verträgliches Ergenis, da SWR in diesen Bereichen generell nur sehr ungenau bestimmt werden können. Insofern kann die Genauigkeit der Messgeräte als ausreichend für diese Art von Messungen eingeschätzt werden.\\
In analoger Weise wird die Abschwächermethode für die Bestimmung des SWR als gut bewertet.\\
Zusammenfassend wird die Durchführung des Versuchs als erfolgreich beurteilt. Es ergeben sich plausible Ergebnisse, sodass die Untersuchung von Mikrowellen auf Hohlleitern mit der vorliegenden Apparatur möglich ist. Auf die Ungenauigkeiten in der Dämpfungsmessung wurde bereits eingegangen; diese scheinen hauptsächlich auf grobe Fehler der Experimentatoren zurückzuführen sein, was durch Wiederholungsmessungen genauer untersucht werden kann.
