\section{Diskussion}
\label{sec:Diskussion}
Insgesamt kann die Durchführung des Versuchs nur bedingt als erfolgreich
bewertet werden.

Bei der Überprüfung der Stabilität liegen die Messwerte nur grob auf den
Ausgleichsfunktionen. Das spiegelt sich auch in den groß aufallenden Fehlern bei
den in den Ausgleichsrechnungen bestimmten Parametern wieder. Die Messwerte folgen
jedoch im Allgemeinen dem Verlauf der theoretisch zu erwartenden Kurven.

Bei der Vermessung der TEM$_{\mathrm{0,0}}$ Mode scheint ein grober Messfehler
unterlaufen zu sein. Obwohl enizig und allein die Linse im Strahlengang des Lasers
stand, lies sich, wie in Abbildung \ref{fig:bild} zu sehen ist, bereits auf dem
Schirm erkennen, dass die Intensität nicht dem theoretisch zu erwartenden gaußförmigen
Verlauf folgt. Das zeigt auch der Verlauf der Messwerte in Abbildung \ref{fig:tem00}.
Zudem ist dort zu erkennen, dass die Theoriekurve nicht sinnvoll an die Messwerte
angepasst werden kann.
Für die TEM$_{\mathrm{0,1}}$ Mode ergibt sich, wie in Abbildung \ref{fig:tem01} zu
sehen ist, ebenfalls nur eine bedingt sinnvolle Ausgleichsfunktion für die Messwerte.
Die Messwerte sind nicht symmetrisch angeordnet, wie es die Theoriekurve verlangen würde.
Das kann darauf zurückgeführt werden, dass der Draht im Experiment nicht perfekt
justiert wurde.

Die Messung der Polarisierung des Lichts zeigt den theoretisch zu erwartenden Verlauf
der Intensität des Lasers in Abhängigkeit vom Winkel des Polarisationsfilters. Wie
in Abbildun \ref{fig:polarisation} zu sehen ist, kann die Theoriekurve sehr gut an
die Messwerte gefittet werden. Es zeigt sich eine Phasenverschiebung von etwa -20°.
Die experimentell bestimmte Wellenlänge stimmt mit $\lambda=\SI{652(15)}{\nano\metre}$
und einer Abweichung von 3.0\,\% sehr gut mit der Theorie überein.

WAS MACHEN DIE RESONATORSPIEGEL?

Phasenverschiebung eventuell durch die Resonatorspiegel.
