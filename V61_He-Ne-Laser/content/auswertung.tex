\section{Auswertung}
\label{sec:Auswertung}

\subsection{Überprüfung der Stabilität}
\label{subsec:stabilitaet}
Die Messwerte für den realisierten Laser mit einem flachen und einem sphärischen
Spiegel befinden sich in Tabelle \ref{tab:plankonkav}. In Abbildung \ref{fig:plankonkav}
sind die gemessenen Stromstärken $I$ gegen die Resonatorlängen $L$ aufgetragen.
Zudem wird eine lineare Ausgleichrechung der Form
\begin{equation*}
  f(L)=a L+b
\end{equation*}
durchgeführt. Für die Parameter $a$ und $b$ ergeben sich die Werte
\begin{align*}
  a&= \SI{2.61(029)}{\micro\ampere\per\centi\per\metre}\,,\\
  b&= \SI{214(18)}{\micro\ampere}\,.
\end{align*}

\begin{figure}
  \centering
  \includegraphics{build/plankonkav.pdf}
  \caption{Darstellung der Messwerte sowie einer gefitteten Ausgleichsfunktion für den
  Resonator aus einem flachen und einem sphärischen Spiegel.}
  \label{fig:plankonkav}
\end{figure}

Die Messwerte für den Resonator aus zwei sphärischen Spiegeln befinden sich in
Tabelle \ref{tab:konkavkonkav}. In Abbildung \ref{fig:konkavkonkav} sind die gemessenen Stromstärken gegen die Resonatorlänge aufgetragen. Außerdem wird eine
Ausgleichsrechnung der Form
\begin{equation*}
  f(L)=a L^2 + b L + c
\end{equation*}
durchgeführt. Es ergeben sich die Parameter
\begin{align*}
  a&= \SI{0.097(017)}{\micro\ampere\per\centi\per\metre\squared}\,, \\
  b&= \SI{-21(4)}{\micro\ampere\per\centi\per\metre}\,, \\
  c&= \SI{1.15(020)e3}{\micro\ampere}\,.
\end{align*}

\begin{figure}
  \centering
  \includegraphics{build/konkavkonkav.pdf}
  \caption{Darstellung der Messwerte sowie einer gefitteten Ausgleichsfunktion für den
  Resonator aus zwei sphärischen Spiegeln.}
  \label{fig:konkavkonkav}
\end{figure}

\subsection{Vermessung der Moden}
\label{subsec:moden}

Die gemessenen Werte für die $TEM_{0,0}$ Mode befinden sich in den Tabellen \ref{tab:tem00a} und \ref{tab:tem00b}. In Abbildung \ref{fig:tem00} sind sie
grafisch dargestellt.

\begin{figure}
  \centering
  \includegraphics{build/tem00.pdf}
  \caption{Darstellung der Messwerte sowie einer gefitteten Ausgleichsfunktion für die $TEM_{0,0}$ Mode.}
  \label{fig:tem00}
\end{figure}

Die Messwerte für die $TEM_{0,1}$ Mode sind in den Tabellen \ref{tab:tem01a} und \ref{tab:tem01b} aufgeführt. In Abbildung \ref{fig:tem01} sind sie grafisch dargestellt.

\begin{figure}
  \centering
  \includegraphics{build/tem01.pdf}
  \caption{Darstellung der Messwerte sowie einer gefitteten Ausgleichsfunktion für die $TEM_{0,1}$ Mode.}
  \label{fig:tem01}
\end{figure}

\subsection{Messung der Polarisierung}
\label{subsec:polarisierung}

In Tabelle \ref{tab:polarisation} befinden sich die Messwerte zur Intensität des
Laserstrahls in Abhängigkeit von dem Winkel des Polarisationsfilters. Diese Werte
sind in Abbildung \ref{fig:polarisation} graphisch dargestellt. Es wird eine Ausgleichsfunktion der Form
\begin{equation*}
  f(\phi)=I_{\text{max}} cos^2(\phi+\delta)
\end{equation*}
durchgeführt. Diese liefert die Parameter
\begin{align*}
  I_{\text{max}}&=\SI{56.02(118)}{\micro\ampere} \,, \\
  \delta&=\SI{919.32(105)} \,.
\end{align*}

\begin{figure}
  \centering
  \includegraphics{build/polarisation.pdf}
  \caption{Darstellung der Messwerte sowie einer gefitteten Ausgleichsfunktion für die Intensität in
  Abhängigkeit vom Winkel des Polarisationsfilters.}
  \label{fig:polarisation}
\end{figure}


\subsection{Bestimmung der Wellenlänge}
\label{subsec:wellenlaenge}
