\newpage
\addsec{Anhang}
\label{sec:Anhang}
\begin{figure}[h!]
  \centering
  \begin{subfigure}{0.49\textwidth}
    \centering
    \includegraphics[width=0.90\textwidth]{build/zyl1.pdf}
    \caption{Am Computer aufgenommenes Frequenzspektrum für einen
    Zylinder. Die am Oszilloskop gemessenen Resonanzfrequenzen sind als rote vertikale
    Linien eingezeichnet.}
    \label{fig:zyl1}
  \end{subfigure}
  \begin{subfigure}{0.49\textwidth}
    \centering
    \includegraphics[width=0.90\textwidth]{build/zyl3.pdf}
    \caption{Am Computer aufgenommenes Frequenzspektrum für drei aneinandergelegte
    Zylinder. Die am Oszilloskop gemessenen Resonanzfrequenzen sind als rote vertikale
    Linien eingezeichnet.}
    \label{fig:zyl3}
  \end{subfigure}
  \begin{subfigure}{0.49\textwidth}
    \centering
    \includegraphics[width=0.90\textwidth]{build/zyl4.pdf}
    \caption{Am Computer aufgenommenes Frequenzspektrum für vier aneinandergelegte
    Zylinder. Die am Oszilloskop gemessenen Resonanzfrequenzen sind als rote vertikale
    Linien eingezeichnet.}
    \label{fig:zyl4}
  \end{subfigure}
  \begin{subfigure}{0.49\textwidth}
    \centering
    \includegraphics[width=0.90\textwidth]{build/zyl5.pdf}
    \caption{Am Computer aufgenommenes Frequenzspektrum für fünf aneinandergelegte
    Zylinder. Die am Oszilloskop gemessenen Resonanzfrequenzen sind als rote vertikale
    Linien eingezeichnet.}
    \label{fig:zyl5}
  \end{subfigure}
  \begin{subfigure}{0.49\textwidth}
    \centering
    \includegraphics[width=0.90\textwidth]{build/zyl6.pdf}
    \caption{Am Computer aufgenommenes Frequenzspektrum für sechs aneinandergelegte
    Zylinder. Die am Oszilloskop gemessenen Resonanzfrequenzen sind als rote vertikale
    Linien eingezeichnet.}
    \label{fig:zyl6}
  \end{subfigure}
  \begin{subfigure}{0.49\textwidth}
    \centering
    \includegraphics[width=0.90\textwidth]{build/zyl7.pdf}
    \caption{Am Computer aufgenommenes Frequenzspektrum für sieben aneinandergelegte
    Zylinder. Die am Oszilloskop gemessenen Resonanzfrequenzen sind als rote vertikale
    Linien eingezeichnet.}
    \label{fig:zyl7}
  \end{subfigure}
\end{figure}

\begin{figure}
  \centering
  \begin{subfigure}{0.49\textwidth}
    \centering
    \includegraphics[width=0.95\textwidth]{build/zyl8.pdf}
    \caption{Am Computer aufgenommenes Frequenzspektrum für acht aneinandergelegte
    Zylinder. Die am Oszilloskop gemessenen Resonanzfrequenzen sind als rote vertikale
    Linien eingezeichnet.}
    \label{fig:zyl8}
  \end{subfigure}
  \begin{subfigure}{0.49\textwidth}
    \centering
    \includegraphics[width=0.95\textwidth]{build/zyl9.pdf}
    \caption{Am Computer aufgenommenes Frequenzspektrum für neun aneinandergelegte
    Zylinder. Die am Oszilloskop gemessenen Resonanzfrequenzen sind als rote vertikale
    Linien eingezeichnet.}
    \label{fig:zyl9}
  \end{subfigure}
  \begin{subfigure}{0.49\textwidth}
    \centering
    \includegraphics[width=0.95\textwidth]{build/zyl10.pdf}
    \caption{Am Computer aufgenommenes Frequenzspektrum für zehn aneinandergelegte
    Zylinder. Die am Oszilloskop gemessenen Resonanzfrequenzen sind als rote vertikale
    Linien eingezeichnet.}
    \label{fig:zyl10}
  \end{subfigure}
  \begin{subfigure}{0.49\textwidth}
    \centering
    \includegraphics[width=0.95\textwidth]{build/zyl11.pdf}
    \caption{Am Computer aufgenommenes Frequenzspektrum für elf aneinandergelegte
    Zylinder. Die am Oszilloskop gemessenen Resonanzfrequenzen sind als rote vertikale
    Linien eingezeichnet.}
    \label{fig:zyl11}
  \end{subfigure}
\end{figure}

\begin{figure}
  \centering
  \begin{subfigure}{0.49\textwidth}
    \centering
    \includegraphics[width=0.95\textwidth]{build/hatom18.pdf}
    \caption{Am Computer aufgenommenes Frequenzspektrum für den Kugelresonator bei
    einem Winkel von $\alpha=180°$}
    \label{fig:hatom18}
  \end{subfigure}
  \begin{subfigure}{0.49\textwidth}
    \centering
    \includegraphics[width=0.95\textwidth]{build/hatom17.pdf}
    \caption{Am Computer aufgenommenes Frequenzspektrum für den Kugelresonator bei
    einem Winkel von $\alpha=170°$}
    \label{fig:hatom17}
  \end{subfigure}
  \begin{subfigure}{0.49\textwidth}
    \centering
    \includegraphics[width=0.95\textwidth]{build/hatom16.pdf}
    \caption{Am Computer aufgenommenes Frequenzspektrum für den Kugelresonator bei
    einem Winkel von $\alpha=160°$}
    \label{fig:hatom16}
  \end{subfigure}
  \begin{subfigure}{0.49\textwidth}
    \centering
    \includegraphics[width=0.95\textwidth]{build/hatom15.pdf}
    \caption{Am Computer aufgenommenes Frequenzspektrum für den Kugelresonator bei
    einem Winkel von $\alpha=150°$}
    \label{fig:hatom15}
  \end{subfigure}
  \begin{subfigure}{0.49\textwidth}
    \centering
    \includegraphics[width=0.95\textwidth]{build/hatom14.pdf}
    \caption{Am Computer aufgenommenes Frequenzspektrum für den Kugelresonator bei
    einem Winkel von $\alpha=140°$}
    \label{fig:hatom14}
  \end{subfigure}
  \begin{subfigure}{0.49\textwidth}
    \centering
    \includegraphics[width=0.95\textwidth]{build/hatom13.pdf}
    \caption{Am Computer aufgenommenes Frequenzspektrum für den Kugelresonator bei
    einem Winkel von $\alpha=130°$}
    \label{fig:hatom13}
  \end{subfigure}
\end{figure}

\begin{figure}
  \centering
  \begin{subfigure}{0.49\textwidth}
    \centering
    \includegraphics[width=0.95\textwidth]{build/hatom12.pdf}
    \caption{Am Computer aufgenommenes Frequenzspektrum für den Kugelresonator bei
    einem Winkel von $\alpha=120°$}
    \label{fig:hatom12}
  \end{subfigure}
  \begin{subfigure}{0.49\textwidth}
    \centering
    \includegraphics[width=0.95\textwidth]{build/hatom11.pdf}
    \caption{Am Computer aufgenommenes Frequenzspektrum für den Kugelresonator bei
    einem Winkel von $\alpha=110°$}
    \label{fig:hatom11}
  \end{subfigure}
  \begin{subfigure}{0.49\textwidth}
    \centering
    \includegraphics[width=0.95\textwidth]{build/hatom10.pdf}
    \caption{Am Computer aufgenommenes Frequenzspektrum für den Kugelresonator bei
    einem Winkel von $\alpha=100°$}
    \label{fig:hatom10}
  \end{subfigure}
  \begin{subfigure}{0.49\textwidth}
    \centering
    \includegraphics[width=0.95\textwidth]{build/hatom9.pdf}
    \caption{Am Computer aufgenommenes Frequenzspektrum für den Kugelresonator bei
    einem Winkel von $\alpha=90°$}
    \label{fig:hatom9}
  \end{subfigure}
  \begin{subfigure}{0.49\textwidth}
    \centering
    \includegraphics[width=0.95\textwidth]{build/hatom8.pdf}
    \caption{Am Computer aufgenommenes Frequenzspektrum für den Kugelresonator bei
    einem Winkel von $\alpha=80°$}
    \label{fig:hatom8}
  \end{subfigure}
  \begin{subfigure}{0.49\textwidth}
    \centering
    \includegraphics[width=0.95\textwidth]{build/hatom7.pdf}
    \caption{Am Computer aufgenommenes Frequenzspektrum für den Kugelresonator bei
    einem Winkel von $\alpha=70°$}
    \label{fig:hatom7}
  \end{subfigure}
\end{figure}

\begin{figure}
  \centering
  \begin{subfigure}{0.49\textwidth}
    \centering
    \includegraphics[width=0.95\textwidth]{build/hatom6.pdf}
    \caption{Am Computer aufgenommenes Frequenzspektrum für den Kugelresonator bei
    einem Winkel von $\alpha=60°$}
    \label{fig:hatom6}
  \end{subfigure}
  \begin{subfigure}{0.49\textwidth}
    \centering
    \includegraphics[width=0.95\textwidth]{build/hatom5.pdf}
    \caption{Am Computer aufgenommenes Frequenzspektrum für den Kugelresonator bei
    einem Winkel von $\alpha=50°$}
    \label{fig:hatom5}
  \end{subfigure}
  \begin{subfigure}{0.49\textwidth}
    \centering
    \includegraphics[width=0.95\textwidth]{build/hatom4.pdf}
    \caption{Am Computer aufgenommenes Frequenzspektrum für den Kugelresonator bei
    einem Winkel von $\alpha=40°$}
    \label{fig:hatom4}
  \end{subfigure}
  \begin{subfigure}{0.49\textwidth}
    \centering
    \includegraphics[width=0.95\textwidth]{build/hatom3.pdf}
    \caption{Am Computer aufgenommenes Frequenzspektrum für den Kugelresonator bei
    einem Winkel von $\alpha=30°$}
    \label{fig:hatom3}
  \end{subfigure}
  \begin{subfigure}{0.49\textwidth}
    \centering
    \includegraphics[width=0.95\textwidth]{build/hatom2.pdf}
    \caption{Am Computer aufgenommenes Frequenzspektrum für den Kugelresonator bei
    einem Winkel von $\alpha=20°$}
    \label{fig:hatom2}
  \end{subfigure}
  \begin{subfigure}{0.49\textwidth}
    \centering
    \includegraphics[width=0.95\textwidth]{build/hatom1.pdf}
    \caption{Am Computer aufgenommenes Frequenzspektrum für den Kugelresonator bei
    einem Winkel von $\alpha=10°$}
    \label{fig:hatom1}
  \end{subfigure}
\end{figure}

\begin{figure}
  \centering
  \includegraphics[width=350pt]{build/hatom0.pdf}
  \caption{Am Computer aufgenommenes Frequenzspektrum für den Kugelresonator bei
  einem Winkel von $\alpha=0°$}
  \label{fig:hatom0}
\end{figure}
