\newpage
\addsec{Anhang}
\label{sec:Anhang}

\begin{figure}
  \centering
  \includegraphics[width=300pt]{build/zyl1.pdf}
  \caption{Am Computer augenommenes Frequenzspektrum für einen
  Zylinder. Die am Oszilloskop gemessenen Resonanzfrequenzen sind als rote vertikale
  Linien eingezeichnet.}
  \label{fig:zyl1}
\end{figure}

\begin{figure}
  \centering
  \includegraphics[width=300pt]{build/zyl3.pdf}
  \caption{Am Computer augenommenes Frequenzspektrum für drei aneinandergelegte
  Zylinder. Die am Oszilloskop gemessenen Resonanzfrequenzen sind als rote vertikale
  Linien eingezeichnet.}
  \label{fig:zyl3}
\end{figure}

\begin{figure}
  \centering
  \includegraphics[width=300pt]{build/zyl4.pdf}
  \caption{Am Computer augenommenes Frequenzspektrum für vier aneinandergelegte
  Zylinder. Die am Oszilloskop gemessenen Resonanzfrequenzen sind als rote vertikale
  Linien eingezeichnet.}
  \label{fig:zyl4}
\end{figure}

\begin{figure}
  \centering
  \includegraphics[width=300pt]{build/zyl5.pdf}
  \caption{Am Computer augenommenes Frequenzspektrum für fünf aneinandergelegte
  Zylinder. Die am Oszilloskop gemessenen Resonanzfrequenzen sind als rote vertikale
  Linien eingezeichnet.}
  \label{fig:zyl5}
\end{figure}

\begin{figure}
  \centering
  \includegraphics[width=300pt]{build/zyl6.pdf}
  \caption{Am Computer augenommenes Frequenzspektrum für sechs aneinandergelegte
  Zylinder. Die am Oszilloskop gemessenen Resonanzfrequenzen sind als rote vertikale
  Linien eingezeichnet.}
  \label{fig:zyl6}
\end{figure}

\begin{figure}
  \centering
  \includegraphics[width=300pt]{build/zyl7.pdf}
  \caption{Am Computer augenommenes Frequenzspektrum für sieben aneinandergelegte
  Zylinder. Die am Oszilloskop gemessenen Resonanzfrequenzen sind als rote vertikale
  Linien eingezeichnet.}
  \label{fig:zyl7}
\end{figure}

\begin{figure}
  \centering
  \includegraphics[width=300pt]{build/zyl8.pdf}
  \caption{Am Computer augenommenes Frequenzspektrum für acht aneinandergelegte
  Zylinder. Die am Oszilloskop gemessenen Resonanzfrequenzen sind als rote vertikale
  Linien eingezeichnet.}
  \label{fig:zyl8}
\end{figure}

\begin{figure}
  \centering
  \includegraphics[width=300pt]{build/zyl9.pdf}
  \caption{Am Computer augenommenes Frequenzspektrum für neun aneinandergelegte
  Zylinder. Die am Oszilloskop gemessenen Resonanzfrequenzen sind als rote vertikale
  Linien eingezeichnet.}
  \label{fig:zyl9}
\end{figure}

\begin{figure}
  \centering
  \includegraphics[width=300pt]{build/zyl10.pdf}
  \caption{Am Computer augenommenes Frequenzspektrum für zehn aneinandergelegte
  Zylinder. Die am Oszilloskop gemessenen Resonanzfrequenzen sind als rote vertikale
  Linien eingezeichnet.}
  \label{fig:zyl10}
\end{figure}

\begin{figure}
  \centering
  \includegraphics[width=300pt]{build/zyl11.pdf}
  \caption{Am Computer augenommenes Frequenzspektrum für elf aneinandergelegte
  Zylinder. Die am Oszilloskop gemessenen Resonanzfrequenzen sind als rote vertikale
  Linien eingezeichnet.}
  \label{fig:zyl11}
\end{figure}
