\newpage
\addsec{Anhang}
\label{sec:Anhang}

\begin{figure}
  \centering
  \includegraphics[width=350pt]{build/zyl1.pdf}
  \caption{Am Computer aufgenommenes Frequenzspektrum für einen
  Zylinder. Die am Oszilloskop gemessenen Resonanzfrequenzen sind als rote vertikale
  Linien eingezeichnet.}
  \label{fig:zyl1}
\end{figure}

\begin{figure}
  \centering
  \includegraphics[width=350pt]{build/zyl3.pdf}
  \caption{Am Computer aufgenommenes Frequenzspektrum für drei aneinandergelegte
  Zylinder. Die am Oszilloskop gemessenen Resonanzfrequenzen sind als rote vertikale
  Linien eingezeichnet.}
  \label{fig:zyl3}
\end{figure}

\begin{figure}
  \centering
  \includegraphics[width=350pt]{build/zyl4.pdf}
  \caption{Am Computer aufgenommenes Frequenzspektrum für vier aneinandergelegte
  Zylinder. Die am Oszilloskop gemessenen Resonanzfrequenzen sind als rote vertikale
  Linien eingezeichnet.}
  \label{fig:zyl4}
\end{figure}

\begin{figure}
  \centering
  \includegraphics[width=350pt]{build/zyl5.pdf}
  \caption{Am Computer aufgenommenes Frequenzspektrum für fünf aneinandergelegte
  Zylinder. Die am Oszilloskop gemessenen Resonanzfrequenzen sind als rote vertikale
  Linien eingezeichnet.}
  \label{fig:zyl5}
\end{figure}

\begin{figure}
  \centering
  \includegraphics[width=350pt]{build/zyl6.pdf}
  \caption{Am Computer aufgenommenes Frequenzspektrum für sechs aneinandergelegte
  Zylinder. Die am Oszilloskop gemessenen Resonanzfrequenzen sind als rote vertikale
  Linien eingezeichnet.}
  \label{fig:zyl6}
\end{figure}

\begin{figure}
  \centering
  \includegraphics[width=350pt]{build/zyl7.pdf}
  \caption{Am Computer aufgenommenes Frequenzspektrum für sieben aneinandergelegte
  Zylinder. Die am Oszilloskop gemessenen Resonanzfrequenzen sind als rote vertikale
  Linien eingezeichnet.}
  \label{fig:zyl7}
\end{figure}

\begin{figure}
  \centering
  \includegraphics[width=350pt]{build/zyl8.pdf}
  \caption{Am Computer aufgenommenes Frequenzspektrum für acht aneinandergelegte
  Zylinder. Die am Oszilloskop gemessenen Resonanzfrequenzen sind als rote vertikale
  Linien eingezeichnet.}
  \label{fig:zyl8}
\end{figure}

\begin{figure}
  \centering
  \includegraphics[width=350pt]{build/zyl9.pdf}
  \caption{Am Computer aufgenommenes Frequenzspektrum für neun aneinandergelegte
  Zylinder. Die am Oszilloskop gemessenen Resonanzfrequenzen sind als rote vertikale
  Linien eingezeichnet.}
  \label{fig:zyl9}
\end{figure}

\begin{figure}
  \centering
  \includegraphics[width=350pt]{build/zyl10.pdf}
  \caption{Am Computer aufgenommenes Frequenzspektrum für zehn aneinandergelegte
  Zylinder. Die am Oszilloskop gemessenen Resonanzfrequenzen sind als rote vertikale
  Linien eingezeichnet.}
  \label{fig:zyl10}
\end{figure}

\begin{figure}
  \centering
  \includegraphics[width=350pt]{build/zyl11.pdf}
  \caption{Am Computer aufgenommenes Frequenzspektrum für elf aneinandergelegte
  Zylinder. Die am Oszilloskop gemessenen Resonanzfrequenzen sind als rote vertikale
  Linien eingezeichnet.}
  \label{fig:zyl11}
\end{figure}

\begin{figure}
  \centering
  \includegraphics[width=350pt]{build/hatom18.pdf}
  \caption{Am Computer aufgenommenes Frequenzspektrum für den Kugelresonator bei
  einem Winkel von $\alpha=180°$}
  \label{fig:hatom18}
\end{figure}

\begin{figure}
  \centering
  \includegraphics[width=350pt]{build/hatom17.pdf}
  \caption{Am Computer aufgenommenes Frequenzspektrum für den Kugelresonator bei
  einem Winkel von $\alpha=170°$}
  \label{fig:hatom17}
\end{figure}

\begin{figure}
  \centering
  \includegraphics[width=350pt]{build/hatom16.pdf}
  \caption{Am Computer aufgenommenes Frequenzspektrum für den Kugelresonator bei
  einem Winkel von $\alpha=160°$}
  \label{fig:hatom16}
\end{figure}

\begin{figure}
  \centering
  \includegraphics[width=350pt]{build/hatom15.pdf}
  \caption{Am Computer aufgenommenes Frequenzspektrum für den Kugelresonator bei
  einem Winkel von $\alpha=150°$}
  \label{fig:hatom15}
\end{figure}

\begin{figure}
  \centering
  \includegraphics[width=350pt]{build/hatom14.pdf}
  \caption{Am Computer aufgenommenes Frequenzspektrum für den Kugelresonator bei
  einem Winkel von $\alpha=140°$}
  \label{fig:hatom14}
\end{figure}

\begin{figure}
  \centering
  \includegraphics[width=350pt]{build/hatom13.pdf}
  \caption{Am Computer aufgenommenes Frequenzspektrum für den Kugelresonator bei
  einem Winkel von $\alpha=130°$}
  \label{fig:hatom13}
\end{figure}

\begin{figure}
  \centering
  \includegraphics[width=350pt]{build/hatom12.pdf}
  \caption{Am Computer aufgenommenes Frequenzspektrum für den Kugelresonator bei
  einem Winkel von $\alpha=120°$}
  \label{fig:hatom12}
\end{figure}

\begin{figure}
  \centering
  \includegraphics[width=350pt]{build/hatom11.pdf}
  \caption{Am Computer aufgenommenes Frequenzspektrum für den Kugelresonator bei
  einem Winkel von $\alpha=110°$}
  \label{fig:hatom11}
\end{figure}
\begin{figure}
  \centering
  \includegraphics[width=350pt]{build/hatom10.pdf}
  \caption{Am Computer aufgenommenes Frequenzspektrum für den Kugelresonator bei
  einem Winkel von $\alpha=100°$}
  \label{fig:hatom10}
\end{figure}

\begin{figure}
  \centering
  \includegraphics[width=350pt]{build/hatom9.pdf}
  \caption{Am Computer aufgenommenes Frequenzspektrum für den Kugelresonator bei
  einem Winkel von $\alpha=90°$}
  \label{fig:hatom9}
\end{figure}

\begin{figure}
  \centering
  \includegraphics[width=350pt]{build/hatom8.pdf}
  \caption{Am Computer aufgenommenes Frequenzspektrum für den Kugelresonator bei
  einem Winkel von $\alpha=80°$}
  \label{fig:hatom8}
\end{figure}

\begin{figure}
  \centering
  \includegraphics[width=350pt]{build/hatom7.pdf}
  \caption{Am Computer aufgenommenes Frequenzspektrum für den Kugelresonator bei
  einem Winkel von $\alpha=70°$}
  \label{fig:hatom7}
\end{figure}

\begin{figure}
  \centering
  \includegraphics[width=350pt]{build/hatom6.pdf}
  \caption{Am Computer aufgenommenes Frequenzspektrum für den Kugelresonator bei
  einem Winkel von $\alpha=60°$}
  \label{fig:hatom6}
\end{figure}

\begin{figure}
  \centering
  \includegraphics[width=350pt]{build/hatom5.pdf}
  \caption{Am Computer aufgenommenes Frequenzspektrum für den Kugelresonator bei
  einem Winkel von $\alpha=50°$}
  \label{fig:hatom5}
\end{figure}

\begin{figure}
  \centering
  \includegraphics[width=350pt]{build/hatom4.pdf}
  \caption{Am Computer aufgenommenes Frequenzspektrum für den Kugelresonator bei
  einem Winkel von $\alpha=40°$}
  \label{fig:hatom4}
\end{figure}

\begin{figure}
  \centering
  \includegraphics[width=350pt]{build/hatom3.pdf}
  \caption{Am Computer aufgenommenes Frequenzspektrum für den Kugelresonator bei
  einem Winkel von $\alpha=30°$}
  \label{fig:hatom3}
\end{figure}

\begin{figure}
  \centering
  \includegraphics[width=350pt]{build/hatom2.pdf}
  \caption{Am Computer aufgenommenes Frequenzspektrum für den Kugelresonator bei
  einem Winkel von $\alpha=20°$}
  \label{fig:hatom2}
\end{figure}

\begin{figure}
  \centering
  \includegraphics[width=350pt]{build/hatom1.pdf}
  \caption{Am Computer aufgenommenes Frequenzspektrum für den Kugelresonator bei
  einem Winkel von $\alpha=10°$}
  \label{fig:hatom1}
\end{figure}

\begin{figure}
  \centering
  \includegraphics[width=350pt]{build/hatom0.pdf}
  \caption{Am Computer aufgenommenes Frequenzspektrum für den Kugelresonator bei
  einem Winkel von $\alpha=0°$}
  \label{fig:hatom0}
\end{figure}
