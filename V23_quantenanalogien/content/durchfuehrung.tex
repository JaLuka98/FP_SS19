\section{Durchführung}
\label{sec:Durchführung}
Die Bestandteile und der grundlegende Aufbau des Versuchs ist in Abbildung \ref{fig:grundlegenderAufbau} zu sehen. Es sind Aluminiumzylinder mit Durchmessern von 12,5, 50 und $\SI{75}{\milli\meter}$ vorhanden. Diese dienen dazu, den eindimensionalen Resonator nachzubilden. Dieser kann zwischen einem Mikrophon und einem Lautsprecher in einer Halterung aufgebaut werden. Für die Untersuchung des Kugelresonators als Analogie zum Wasserstoffatom stehen zwei Halbkugeln zur Verfügung. In einer der beiden befindet sich ein Mikrophon und in der anderen der Lautsprecher.\\
SKALA???????????\\
Eine mögliche Schaltung des Versuchs ist in \ref{fig:schaltung} schematisch dargestellt. Konkret wird dabei das Signal des Mikrophons am Oszilloskop untersucht. Inbesondere die Bestandteile des Steuergeräts sind dort sichtbar.
Dieses verfügt über einen Eingang für das Mikrophon, dessen Signal einstellbar abgeschwächt werden und dann ausgegeben werden kann. Ein Frequenz-zu-Spannungs-Konverter ist vorhanden, sodass das Frequenzspektrum mit einem Oszilloskop oder mit einem Computer aufgenommen werden kann. Ein Sinusgenerator dient zur Anregung der Resonanzen.\\
\subsection{Durchführung vorbereitender Experimente}
\label{subsec:durchfuehrungVorbereitender}
Um sich mit dem Aufbau und dem Messprinzip vertraut zu machen, werden zwei Teilexperimente durchgeführt.\\
Im ersten Teilexperiment wird gemäß \ref{fig:schaltung} aufgebaut und die Signale am Oszilloskop untersucht. Ein Kanal des Oszilloskops stellt die Lautsprecherspannung und der andere die Mikrophonspannung dar. Der nun beschriebene Ablauf wird für 1,2,3,...,12 Zylinder wiederholt. Die Generatorfrequenz wird auf $\SI{6.75}{\kilo\hertz}$ geregelt. Dann wird sie erhöht, bis eine Resonanz am Oszilloskop ersichtlich ist. Dies ist genau dann der Fall, wenn die Amplitude im Vergleich zur Umgebung der Resonanzfrequenz maximal ist. Diese Frequenz wird gemeinsam mit der Amplitude UND WAS SONST NOCH SO? aufgezeichnet.\\
Im zweiten Teilexperiment
