\section{Diskussion}
\label{sec:Diskussion}

Der Versuch lässt sich insgesamt nur teilweise als erfolgreich bewerten. Die bestimmte
Schallgeschwindigkeit ist mit $\SI{339.7(7)}{\meter\per\second}$ und einer relativen Abweichung zum Theoriewert
von $-0{,}85\%$ sehr genau.
Der in den Abbildungen \ref{fig:zyl2}, \ref{fig:zyl12} und \ref{fig:zyl1} bis \ref{fig:zyl11}
dargestellte Vergleich zwischen den Messungen am Zylinderresonator mithilfe des Oszilloskops und denen mithilfe des
Computers zeigt jedoch Unstimmigkeiten. Für kurze Resonatoren stimmen die Messwerte recht
gut überein, für große Resonatorlängen sind jedoch starke Abweichungen zu erkennen. Der Grund
dafür liegt vermutlich im falschen Betrieb des Lautsprechers und des Mikrofons. Wegen großer
Probleme bei der Inbetriebnahme dieser Instrumente wurde vermutlich ohne externe Soundkarte
gemessen, sodass Lautsprecher und Mikrofon nicht ordnungsgemäß gearbeitet haben.
Auffällig ist, dass die mit dem Computer gemessenen Werte für die Resonanzfrequenzen
nicht äquidistant sind. Das ist ein Widerspruch zu der linearen Dispersion und weist auf
Messfehler hin.

Für den Kugelresonator hingegen zeigt sich eine sehr gute Übereinstimmung zwischen der Messung
am Oszilloskop und der am Computer (vgl. Abbildung \ref{fig:hatomalles}). Die Messwerte für die
Amplitude in Abhängikeit vom Verschiebungswinkel $\alpha$ des Lautsprechers und des Mikrophons
zeigen nur tendenziell eine Analogie zu den gezeigten Theoriekurven. Die Abweichungen sind
jedoch zu groß um eine Aussage darüber treffen zu können, ob die gemessenen Werte
tatsächlich dem Verlauf dieser Kurve folgen. Möglicherweise sind auch die Funktionen
für die Theoriekurven nicht passend gewählt.

In Bezug auf die Quantenanalogien konnte das diskrete Frequenzspektrum des Zylinderresonators, welches in Analogie zu dem diskreten Energiespektrum des eindimensionalen Potenzialtopfes steht, gemessen werden. Zudem konnte
eine grobe Anordnung der Messwerte auf den Graphen der Legendrepolynome gezeigt werden. Dies zeigt, dass die Kugelfächenfunktionen des Kugelresonators denen des
Wasserstoffatoms entsprechen könnten. Um aussagekräftige Ergebnisse über Quantenanalogien treffen zu können müssten allerdings mehr und insbesondere genauere Messungen durchgeführt werden.
