\section{Auswertung}
\label{sec:Auswertung}

\subsection{Vorbereitende Experimente}
\label{subsec:vorbereitung}
Zur Auswertung der ersten Messreihe wird die Differenz der gemessenen
Resonanzfrequenzen doppelt logarithmisch gegen die Anzahl der verwendeten Zylinder
aufgetragen. Die zugrundeliegenden Daten befinden sich in Tabelle \ref{tab:zyl}.

\begin{table}[htp]
	\begin{center}
    \caption{Messwerte zum Zylinderresonator und daraus berechnete Werte.}
    \label{tab:zyl}
		\begin{tabular}{ccccc}
		\toprule
			{Zylinderanzahl} & {Resonatorlänge/cm} & {$f_1/$Hz} & {$f_2/$Hz} & {$\Delta f/$Hz}\\
			\midrule
			1 & 5 & 6880 & 10280 & 3400\\
			2 & 10 & 6900 & 8610 & 1710\\
			3 & 15 & 6910 & 8060 & 1150\\
			4 & 20 & 6910 & 7770 & 860\\
			5 & 25 & 6920 & 7610 & 690\\
			6 & 30 & 6910 & 7490 & 580\\
			7 & 35 & 6910 & 7410 & 500\\
			8 & 40 & 6910 & 7340 & 430\\
			9 & 45 & 6920 & 7300 & 380\\
			10 & 50 & 6920 & 7260 & 340\\
			11 & 55 & 6920 & 7230 & 310\\
			12 & 60 & 6920 & 7210 & 290\\
		\bottomrule
		\end{tabular}
	\end{center}
\end{table}

Gemäß Gleichung \eqref{eqn:stehendeWelle} ist ein Zusammenhang der Form $f_n = nc/(2 L)$
für die $n$-te Resonanzfrequenz zu erwarten. Dementsprechend gilt für die Resonanzfrequenzdifferenz
\begin{equation}
	\Delta f = f_{n+1} - f_n = \frac{1}{2L} c\,.
	\label{eqn:deltaf}
\end{equation}
Dabei ist zu beachten, dass die niedrigere gemessene Resonanzfrequenz als $f_n$ und die höhere als $f_{n+1}$ anzusetzen ist, da ihre Ordnungen unbekannt sind.\\
Es wird eine lineare Ausgleichrechnung der Form $f(x)=ax+b$ durchgeführt, wobei die $x$-Werte bereits als $1/(2L)$ gegeben werden, sodass bei perfekter Messung $a$ der Schallgeschwindigkeit $c$ entspricht und $b$ null ist.\\
Die Berechnungen werden in Python 3.7.1, unterstützt durch das Paket NumPy \cite{numpy}, durchgeführt.
Für die Ausgleichsrechnung wird SciPy \cite{scipy} verwendet. Die Abbildungen werden mit matplotlib \cite{matplotlib} erstellt. Es ergeben sich die Parameter
\begin{align*}
  a&=\SI{339.7(7)}{\meter\per\second}\,, \\
  b&=\SI{8.2(25)}{\Hz}\,.
\end{align*}
In Abbildung \ref{fig:roehre} sind die Messdaten sowie die Ausgleichsrechnung grafisch
dargestellt.
Der Literaturwert der Schallgeschwindigkeit beträgt $\SI{342.60}{\meter\per\second}$ \cite{Schallgeschwindigkeit}, sodass sich eine relative Abweichung von $-0{,}85\%$ ergibt.

\begin{figure}
  \centering
  \includegraphics[width=\textwidth]{build/roehre.pdf}
  \caption{Grafische Darstellung der Messdaten und der zugehörigen Ausgleichsrechnung
  zur Bestimmung der Schallgeschwindigkeit.}
  \label{fig:roehre}
\end{figure}

Nun sollen die Messungen am Oszilloskop mit denen am Computer verglichen werden.
Die Messdaten für die am Oszilloskop gemessenen Resonanzfrequenzen sind in Tabelle
\ref{tab:oszi} augeführt.

\begin{table}[htp]
	\begin{center}
    \caption{Am Oszilloskop aufgenommene Messwerte für die Resonanzfrequenzen. Der
    Index kennzeichnet die Anzahl der für den Resonator verwendeten Zylinder.}
    \label{tab:oszi}
    \tiny
		\begin{tabular}{cccccccccccc}
		\toprule
			{$f_1$/Hz} & {$f_2$/Hz} & {$f_3$/Hz} & {$f_4$/Hz} & {$f_5$/Hz} & {$f_6$/Hz} & {$f_7$/Hz} & {$f_8$/Hz} & {$f_9$/Hz} & {$f_{10}$/Hz} & {$f_{11}$/Hz} & {$f_{12}$/Hz}\\
			\midrule
			6686 & 5330 & 4503 & 5197 & 4849 & 5236 & 5042 & 4732 & 4639 & 4849 & 4733 & 4616\\
			10173 & 6918 & 5786 & 6089 & 5585 & 5997 & 5430 & 5197 & 4965 & 5313 & 5081 & 4888\\
			13545 & 8507 & 6790 & 6941 & 6305 & 6283 & 5934 & 5663 & 5489 & 5546 & 5391 & 5236\\
			{-} & 10251 & 7990 & 7794 & 6903 & 6864 & 6399 & 6050 & 5779 & 5972 & 5701 & 5469\\
			{-} & 11956 & 9190 & 8608 & 7600 & 7445 & 6903 & 6438 & 6205 & 6205 & 6050 & 5779\\
			{-} & 13739 & 10320 & 9383 & 8453 & 8026 & 7368 & 6941 & 6631 & 6593 & 6283 & 6011\\
			{-} & {-} & 11560 & 10351 & 8956 & 8569 & 7949 & 7445 & 7019 & 6941 & 6631 & 6321\\
			{-} & {-} & 12560 & {-} & 9663 & 9111 & 8259 & 7794 & 7406 & 7329 & 6941 & 6631\\
			{-} & {-} & {-} & {-} & 10351 & 9770 & 8995 & 8220 & 7755 & 7639 & 7251 & 6941\\
			{-} & {-} & {-} & {-} & {-} & 10390 & 9344 & 8569 & 8065 & 7949 & 7561 & 7174\\
			{-} & {-} & {-} & {-} & {-} & {-} & 9809 & 8995 & 8491 & 8298 & 7910 & 7484\\
			{-} & {-} & {-} & {-} & {-} & {-} & 10429 & {-} & {-} & 8646 & {-} & {-}\\
		\bottomrule
		\end{tabular}
	\end{center}
\end{table}

In den Abbildungen \ref{fig:zyl2} und \ref{fig:zyl12} sind die mit dem Computer gemessenen
Spektren dargestellt. Die roten senkrechten Linien zeigen die Frequenzen, bei denen
auch am Oszilloskop eine Resonanz gemessen wurde. Beispielhaft ist hier ein Spektrum
gezeigt, bei dem beide Messungen gut übereinstimmen und eines, bei denen am Oszilloskop
viel mehr Resonanzen gemessen wurden als am PC. Die restlichen Abbildungen zu dieser
Messreihe befinden sich im Anhang.

\begin{figure}
  \centering
  \includegraphics[width=\textwidth]{build/zyl2.pdf}
  \caption{Am Computer augenommenes Frequenzspektrum für zwei aneinandergelegte
  Zylinder. Die am Oszilloskop gemessenen Resonanzfrequenzen sind als rote vertikale
  Linien eingezeichnet.}
  \label{fig:zyl2}
\end{figure}
\begin{figure}
  \centering
  \includegraphics[width=\textwidth]{build/zyl12.pdf}
  \caption{Am Computer augenommenes Frequenzspektrum für zwölf aneinandergelegte
  Zylinder. Die am Oszilloskop gemessenen Resonanzfrequenzen sind als rote vertikale
  Linien eingezeichnet.}
  \label{fig:zyl12}
\end{figure}


Allgemein lässt sich feststellen, dass die Ungenauigkeiten mit steigender Zylinderanzahl
bzw. wachsender Röhrenlänge zunehmen. Für wenige Zylinder, also kurze Röhren, stimmen
die beiden Messungen jedoch grob überein.

\subsection{Das Wasserstoffatom}
\label{subsec:hatom}

Das Frequenzspektrum des Kugelresonators ist in Abbildung \ref{fig:hatomalles} dargestellt.
Auch hier sind in rot die Resonanzen dargestellt, die auch mit dem Oszilloskop
gemessen wurden. Die Messwerte befinden sich in Tabelle \ref{fig:oszires}.

\begin{table}[htp]
	\begin{center}
    \caption{Am Oszilloskop gemessene Resonanzfrequenzen für den Kugelkondensator.}
		\label{tab:oszires}
		\begin{tabular}{c}
		\toprule
			{$f_\symup{res}$/Hz}\\
			\midrule
			2394\\
			3673\\
			5029\\
			6269\\
			6618\\
			7431\\
			8051\\
			8671\\
			9446\\
			9756\\
		\bottomrule
		\end{tabular}
	\end{center}
\end{table}

Es zeigt sich eine gute Übereinstimmung.

\begin{figure}
  \centering
  \includegraphics[width=\textwidth]{build/hatomalles.pdf}
  \caption{Am Computer augenommenes Frequenzspektrum für den Kugelresonator.
  Die am Oszilloskop gemessenen Resonanzfrequenzen sind als rote vertikale
  Linien eingezeichnet.}
  \label{fig:hatomalles}
\end{figure}


Die Frequenzsprektren für die Messung der Druckamplituden in Abhängigkeit vom
Drehwinkel $\alpha$ befinden sich im Anhang. Aus den zugehörigen Daten werden
die Amplituden ausgelesen. Die Daten sind in Tabelle \ref{tab:alpha} dargestellt.

\begin{table}[htp]
	\begin{center}
    \caption{Messwerte für drei Peaks bei $f_1=2317$\,Hz, $f_2=3700$\, Hz und
    $f_3=4981$\,Hz in Abhängigkeit vom Verschiebungswinkel $\alpha$.}
    \label{tab:alpha}
		\begin{tabular}{cccc}
		\toprule
			{$\alpha$/°} & {$A_1$} & {$A_2$} & {$A_3$}\\
			\midrule
			0 & 1,11 & 37,40 & 1,43\\
			10 & 0,95 & 14,94 & 1,11\\
			20 & 1,04 & 40,56 & 1,62\\
			30 & 1,26 & 40,60 & 0,98\\
			40 & 1,66 & 33,81 & 1,43\\
			50 & 1,14 & 37,10 & 0,86\\
			60 & 2,53 & 35,19 & 1,37\\
			70 & 1,16 & 11,24 & 0,57\\
			80 & 19,07 & 0,68 & 0,64\\
			90 & 26,35 & 0,65 & 0,72\\
			100 & 26,96 & 0,62 & 0,62\\
			110 & 30,94 & 0,87 & 0,73\\
			120 & 31,72 & 33,94 & 0,73\\
			130 & 32,46 & 35,39 & 1,10\\
			140 & 31,93 & 35,88 & 24,25\\
			150 & 30,20 & 42,94 & 39,94\\
			160 & 34,01 & 31,98 & 32,04\\
			170 & 27,65 & 36,20 & 43,90\\
			180 & 27,32 & 15,98 & 39,36\\
		\bottomrule
		\end{tabular}
	\end{center}
\end{table}

Nun wird die Amplitude in einem Polarplot gegen den Verschiebungswinkel aufgetragen.
Außerdem wird eine Theoriekurve gemäß der zugeordneten Legendrepolynome eingezeichnet.
Das Ergebnis ist in den Abbildungen \ref{fig:polar1}, \ref{fig:polar2} und
\ref{fig:polar3} zu sehen. Die verwendeten zugeordneten Legendrepolynome sind
\begin{align*}
	P_{1,0}(x)=x \,,\\
	P_{2,0}(x)=\frac{1}{2}(3x^2-1) \,,\\
	P_{3,0}(x)=\frac{x}{2}(5x^2-3) \,.
\end{align*}
\begin{figure}
  \centering
  \includegraphics[width=\textwidth]{build/polar1.pdf}
  \caption{Polarplot der Amplitude des Peaks bei $f=2317$\,Hz in Abhängigkeit vom
  Winkel $\alpha$ und zugehörige Theoriekurve $P_{1,0}$.}
  \label{fig:polar1}
\end{figure}
\begin{figure}
  \centering
  \includegraphics[width=\textwidth]{build/polar2.pdf}
  \caption{Polarplot der Amplitude des Peaks bei $f=3700$\,Hz in Abhängigkeit vom
  Winkel $\alpha$ und zugehörige Theoriekurve $P_{2,0}$.}
  \label{fig:polar2}
\end{figure}
\begin{figure}
  \centering
  \includegraphics[width=\textwidth]{build/polar3.pdf}
  \caption{Polarplot der Amplitude des Peaks bei $f=4981$\,Hz in Abhängigkeit vom
  Winkel $\alpha$ und zugehörige Theoriekurve $P_{3,0}$.}
  \label{fig:polar3}
\end{figure}
