\section{Auswertung}
\label{sec:Auswertung}
Zur Auswertung der ersten Messreihe wird die Differenz der gemessenen
Resonanzfrequenzen doppelt logarithmisch gegen die Anzahl der verwendeten Zylinder
aufgetragen. Die zugrundeliegenden Daten befinden sich in Tabelle (TABELLE EINFÜGEN).
Außerdem wird mit den bereits logarithmierten Daten eine Ausgleichsrechnung der
Form
\begin{align*}
  f(x)=ax+b
\end{align*}
durchgeführt. Dafür wird (NUMPY UND SO EINFÜGEN) verwendet. Es ergeben sich die
Fit-Parameter
\begin{align*}
  a&=\SI{-0.996(004)}{\Hz} EINHEIT?!\,, \\
  b&=\SI{8.138(007)}{\Hz}\,.
\end{align*}
In Abbildung \ref{fig:roehre} sind die Messdaten sowie die Ausgleichsrechnung grafisch
dargestellt.

\begin{figure}
  \centering
  \includegraphics[width=\textwidth]{build/roehre.pdf}
  \caption{Grafische Darstellung der Messdaten und der zugehörigen Ausgleichsrechnung
  zur Bestimmung der Schallgeschwindigkeit.}
  \label{fig:roehre}
\end{figure}

Aus den Fit-Parametern wird nun die Schallgeschwindigkeit gemäß (JA, WIE DENN NUN?)
bestimmt.


Nun sollen die Messungen am Oszilloskop mit denen am Computer verglichen werden.
Die Messdaten für die am Oszilloskop gemessenen Resonanzfrequenzen sind in Tabelle
(TABELLE EINFÜGEN) augeführt.
In den Abbildungen \ref{fig:zyl2} und \ref{fig:zyl12} sind die mit dem Computer gemessenen
Spektren dargestellt. Die roten senkrechten Linien zeigen die Frequenzen, bei denen
auch am Oszilloskop eine Resonanz gemessen wurde. Beispielhaft ist hier ein Spektrum
gezeigt, bei dem beide Messungen gut übereinstimmen und eines, bei denen am Oszilloskop
viel mehr Resonanzen gemessen wurden als am PC. Die restlichen Abbildungen zu dieser
Messreihe befinden sich im Anhang.

\begin{figure}
  \centering
  \includegraphics[width=\textwidth]{build/zyl2.pdf}
  \caption{Am Computer augenommenes Frequenzspektrum für zwei aneinandergelegte
  Zylinder. Die am Oszilloskop gemessenen Resonanzfrequenzen sind als rote vertikale
  Linien eingezeichnet.}
  \label{fig:zyl2}
\end{figure}
\begin{figure}
  \centering
  \includegraphics[width=\textwidth]{build/zyl12.pdf}
  \caption{Am Computer augenommenes Frequenzspektrum für zwölf aneinandergelegte
  Zylinder. Die am Oszilloskop gemessenen Resonanzfrequenzen sind als rote vertikale
  Linien eingezeichnet.}
  \label{fig:zyl12}
\end{figure}


Allgemein lässt sich feststellen, dass die Ungenauigkeiten mit steigender Zylinderanzahl
bzw. wachsender Röhrenlänge zunehmen. Für wenige Zylinder, also kurze Röhren, stimmen
die beiden Messungen jedoch grob überein.

Das Frequenzspektrum des Kugelresonators ist in Abbildung \ref{fig:hatomalles} dargestellt.
Auch hier sind in rot die Resonanzen dargestellt, die auch mit dem Oszilloskop
gemessen wurden. Es zeigt sich eine gute Übereinstimmung.

\begin{figure}
  \centering
  \includegraphics[width=\textwidth]{build/hatomalles.pdf}
  \caption{Am Computer augenommenes Frequenzspektrum für den Kugelresonator.
  Die am Oszilloskop gemessenen Resonanzfrequenzen sind als rote vertikale
  Linien eingezeichnet.}
  \label{fig:hatomalles}
\end{figure}


Die Frequenzsprektren für die Messung der Druckamplituden in Abhängigkeit vom
Drehwinkel $\alpha$ befinden sich im Anhang. Aus den zugehörigen Daten werden
die Amplituden ausgelesen. Die Daten sind in Tabelle (REFERENZ) dargestellt.
