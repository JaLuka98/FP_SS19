\section{Aufbau und Durchführung}
\label{sec:aufbauUndDurchfuehrung}

Im Versuch wird der in Abbildung \ref{fig:aufbau} dargestellte Versuchsaufbau verwendet.
Die verwendeten Halbleiter sind für infrarotes Licht durchlässig, sodass das Experiment
darauf ausgerichtet werden muss. Daher werden als Lichtquelle Halogenlampen verwendet,
deren Spektrum zu großen Teilen aus Infrarotlicht besteht.

\begin{figure}
  \centering
  \includegraphics[width=\textwidth]{data/aufbau.png}
  \caption{Schematische Darstellung des Versuchsaufbaus. \cite{anleitung}}
  \label{fig:aufbau}
\end{figure}

Nachdem das Licht durch eine Konensorlinse gebündelt wurde, fällt es auf einen
sog. Lichtzerhacker. Dies ist ein rotierendes Rad, das das zuvor zeitlich kontinuierlich
vorliegende Licht in kurze Lichtpulse zerhackt. Danach fällt es auf einen Polarisator.
In diesem Versuch wird dafür ein Glan-Thompson-Prisma verwendet. Dieses besteht aus
doppelbrechendem Material. Dadruch wird der zunächst unpolarisierte Strahl linear
polarisiert. Dieser linear polarisierte Strahl trifft dann auf die Probe, die ich
im magnetischen Feld eienes Elektromagneten befindet. Darauf folgt ein Interferenzfilter,
in dem eine bestimmte Lichtfrequenz herausgefiltert wird.

Diese trifft auf ein zweites Glan-Thompson-Prisma, das hier als Analysator dient.
Aus dem Prisma treten zwei senkrecht zueinander polarisierte Strahlen aus, die durch zwei
Photowiderstände gemessen werden. Die an den Photowiderständen
gemessenen Spannungen werden auf einen Differenzverstärker gegeben. Die Ausgangsspannung
desselben ist dann proportional zur Differenz der beiden gemessenen Spannungen. Sie
ist genau dann null, wenn beide Wellen identisch sind. Die Frequenz des Differenzverstärkers
wird  mithilfe des Selektivverstärkers auf die des Zerhackers abgestimmt. Die Ausgangsspannung
wird auf ein Oszilloskop gegeben.

Zur Messung des Winkels $\theta$ wird bei maximalem magnetischen Feld die Lichtentensität
der beiden Strahlen auf den gleichen Wert einstellt. Danach wird das erste Glan-Thompson-Prisma
gedreht und die Ausgangsspannung damit auf ein Minimum geregelt. Der entsprechende
Winkel wird notiert. Danach wird das magnetische Feld umgepolt und es wird erneut
durch Drehung des ersten Glan-Thompson-Prismas ein Minimum der Ausgangsspannung gesucht,
dessen Winkel notiert wird. Der Winkel $\theta$ lässt sich dann durch den Zusammenhang
\begin{equation}
  \theta  = \frac{1}{2} (\theta_r - \theta_l)
\label{eqn:drehwinkel}
\end{equation}
bestimmen.

Mithilfe der beschriebenen Apparatur und des erklärten Verfahrens werden im Versuch zwei dotierte
und eine hochreine Galliumarsenidprobe untersucht. Um später die effektive Masse berechnen zu
können, wird außerdem das magnetische Feld am Ort der Probe mit einer Hall-Sonde gemessen.
