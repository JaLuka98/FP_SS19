\section{Diskussion}
\label{sec:Diskussion}

Die Messungen mit beiden Proben sollten das gleiche Ergebnis liefern, da die effektive Masse der Leitungselektronen nicht von der Dotierung in einer Hilfsprobe abhängen kann. Im Rahmen der Messungenauigkeit sollten beide Werte also innerhalb ihrer gemeinsamen Unsicherheiten kompatibel sein, was hier nicht der Fall ist. Im Allgemeinen schränkt eine Reihe von Ungenauigkeiten die Aussagekraft der Ergebnisse ein. Die Erwärmung der Spulen durch den Gleichstrom ist ein in diesem Versuchsaufbau unvermeidbarer Effekt. Dadurch, dass erst am Ende des Versuchs das Magnetfeld vermessen wurde, unterschätzt der gemessene Wert wahrscheinlich die mittlere Flussdichte über den Versuch hinweg ein wenig. Eine Änderung des Versuchsaufbaus wäre nötig, um die magnetische Flussdichte während des Versuchs regelmäßg zu messen. Dann wäre es möglich, eine mehrdimensionale Ausgleichsrechnung unter Berücksichtigung eventueller Unsicherheiten durchzuführen. Die Herstellerangaben für die durchgelassene Wellenlänge durch die Interferenzfilter ist nicht fehlerbehaftet. Der Winkelmesser war sehr fein und ein genaues, der Skala angemessenes Ablesen erscheint mit dem menschlichen Auge nicht möglich. Ein statistischer Fehler auf solche Ungenauigkeiten im Ablesen wurden ebenso nicht berücksichtigt.
Gemeinsam mit einem von der Wellenlänge abhängenden Brechungsindex könnten all diese zusätzlichen Abhängigkeiten und Unsicherheiten unter gewissem Mehraufwand in einer mehrdimensionalen Ausgleichsrechnung umgesetzt werden, um eine konsisterente Abschätzung der Unsicherheit und eine genauere Ausgleichsrechnung zu ermöglichen. Es besteht die Vermutung, dass dann auch die Ergebnisse, die mithilfe der beiden dortierten Proben gewonnen wurden, innerhalb ihrer Standardabweichungen miteinander kompatibel sind.

Zu möglichen systematischen Fehlern gehört, dass das Regeln der Differenzspannung auf ein Minimum mit Amplitude nahe Null aufgrund von Störspannungen und Rauscheffekten nicht möglich war, sodass stets Intervalle mit einer kleinen Breite für die Winkel $\theta_{1,2}$ in Frage kommen würden. Die Abschätzung dieses Fehlers ist nicht geschehen. Des Weiteren ist die Justage der Apparatur nicht ohne systematische Fehler möglich, die schwierig zu quantifizieren sind. Weitere elektronische Unterstützung in diesem Versuch könnte helfen, Teile der beschriebenen Fehlerquellen zu berücksichtigen bzw. zu mitigieren.

Ingesamt weichen die beiden experimentell bestimmten effektiven Massen vom Literaturwert mit $10{,}09\%$ und $17.91\%$ deutlich ab. Im Rahmen der in diesem Versuch erreichbaren Genaugigkeiten ist dieser Versuch dennoch als erfolgreich zu bewerten, insbesondere da das Prinzip der Faraday-Rotation recht anschaulich nachvollzogen werden kann.
