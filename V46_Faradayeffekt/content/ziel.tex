\section{Ziel}
\label{sec:Ziel}
In diesem Versuch soll anhand der Faraday-Rotiation die effektive Masse der
Elektronen in Galliumarsenid bestimmt werden. Dafür wird zunächst der Zusammenhang
zwischen messbaren Größen und der effektiven Masse der Elektronen im Festkörper
erläutert und anschließend ein entsprechendes Messverfahren beschrieben. In diesem
wird der durch den Faraday-Effekt entstehende Drehwinkel für
zwei dotierte und eine hochreine Probe bestimmt. Außerdem wird das magnetische
Feld gemessen. Mithilfe dieser Daten lässt sich am Ende die effektive Masse
der Elektronen in Galliumarsenid bestimmen.
